
\begin{abstract}
    \thispagestyle{plain}
    \pagenumbering{roman}
    \setcounter{page}{2}
    \phantomsection\addcontentsline{toc}{chapter}{\abstractname}
    This final year project report investigates the implementation of an innovative communication system between drones and ground sensors for road traffic surveillance. Addressing the limitations of traditional traffic monitoring systems, the integration of Unmanned Aerial Vehicles (UAVs) offers promising solutions for more efficient traffic management.

    The document begins with a thorough exploration of UAV architectures (multi-rotors, fixed-wing, hybrid designs) and their operational characteristics (weight, endurance, range, altitude). It then examines UAV-specific communication protocols such as MAVLink, UranusLink, and UAVCAN, highlighting their strengths and weaknesses, particularly in terms of security and resource efficiency. Various swarm communication architectures (centralized, decentralized single-group, multi-group, and multi-layer) and associated routing protocols (topology-based, geographic/position-based, or swarm intelligence-based) are also detailed, emphasizing their impact on scalability and robustness.
    
    A significant portion of the work is dedicated to integrating Artificial Intelligence (AI) and Machine Learning (ML) techniques to optimize UAV performance. This includes enhancing trajectory planning, mission scheduling, perception, and feature extraction through supervised, unsupervised, and reinforcement learning methods (such as Deep Q-Networks and Deep Deterministic Policy Gradient). The report also addresses crucial security challenges in UAV networks, identifying threats (eavesdropping, jamming, man-in-the-middle, replay attacks, backdoor, denial of service) and presenting existing solutions to ensure communication integrity and confidentiality.
    
    Finally, the document presents and analyzes six distinct UAV-based road traffic monitoring methods, ranging from Airborne Traffic Surveillance Systems (ATSS) to 5G integration, cooperative surveillance, emergency vehicle routing assistance systems, and collaborative hotspot selection. These case studies highlight the potential of UAVs to provide real-time, adaptive, and scalable traffic management solutions.
    
    In conclusion, this work emphasizes that despite significant advancements, challenges persist regarding operational endurance, communication stability, and regulatory frameworks. Continued innovation in energy systems, network protocols, and policy development will be critical for the widespread adoption of UAVs in future intelligent transportation infrastructures.
    %%
    \\ [2cm]
    %%
    \rule{\linewidth}{1pt}

    \textbf{Keywords --- } 
    \rule{\linewidth}{1pt}
\end{abstract}


% \begin{otherlanguage}{french}
%     \begin{abstract}
%         \thispagestyle{plain}
%         \pagenumbering{roman}
%         \setcounter{page}{1}
%         \phantomsection\addcontentsline{toc}{chapter}{\abstractname}

%         Ceci est un résumé en français.

%         %%
%         \ \\[2cm]
%         %%
%         \rule{\linewidth}{1pt}

%         \textbf{Mots clés --- } Aphasie de Broca, Transformeur.\\
%         \rule{\linewidth}{1pt}
%     \end{abstract}

% \end{otherlanguage}



% \renewcommand{\abstractname}{\RL{مـلـخـص}}
% \afterpage{
%     \newgeometry{top=-2cm}
%     \begin{abstract}
%         \thispagestyle{plain}
%         \pagenumbering{roman}
%         \setcounter{page}{3}
%         \phantomsection\addcontentsline{toc}{chapter}{\textRL{مـلـخـص}}
%         \begin{RLtext}
%             الـحـبـسـة إضـطـرابٌ لـغـوي نـاتـج عن تـلـف فـي الـدمـاغ، غـالـبـا نـتـيـجـة سـكـتـة دمـاغـيـة.
%             حـبـسـة بـروكـا حـبـسـة تـنـتـج عـن إصـابـة فـي مـنـطـقـة بـروكـا،
%             وهـي مـنـطـقـة فـي الـفـص الجـبـهي الأيـسـر للـدمـاغ تـعـنـى بـإنـتـاج الكـلام.
%             قـد يـجـد الـمـصـاب بـحـبـسـة بـروكـا صـعـوبـة فـي تـكـويـن الـجـمـل والـكـلـمـات،
%             إلا أنـه عـادة يـفـهـم مـا يـقـال.
%             تـرتـبـط هـذه الحـبـسـة بـتـدنـي مـسـوى الـعيـش وارتـفتاع  خـطر الاكـتـئـاب والانـتـحـار.

%             عـلاج الـنـطق هـو أكـثـر الـعـلاجـات وصـفـا للمـصـابـيـن بـحـبـسـة بـروكـا.
%             رغـم نـجـاعـتـه، فـهو يـظـل مـكـلـفـا للـوقـت والـمـال والـجـهـد،
%             مـا يـحـول دون تـوفـره لعـدد كـبـيـر مـمـن يـحـتـاجـونـه.

%             تـوظـيـف تـقـنـيـات مـعـالـجـة اللـغـة الـطـبـيـعـيـة لـتـحـسـيـن حـيـاة الـمـصـابـيـن بـحـبـسـة بـروكـا
%             مـجـال بـحـث حـظي بـاهـتـمـام الـعـديـد مـن الـبـاحـثـيـن فـي الأعـوام الأخـيـرة.

%             فـي مـشـروع الـتـخـرج هـذا، نـهـتـم بـاسـتـعـمـال الـتـرجـمـة الآلـية والـتـعرف الآلـي عـلى الكـلام
%             لتـأديـة جـزء مـن عـلاج الـنـطق لـحـبـسـة بـروكـا أوتـومـاتـيـكـيـا.
%             مـن أجـل ذلك، نـعـرض دراسـة بـيـبـلـيـوغـرافـيـة
%             نـعـرف فـيـهـا بـحـبـسـة بـروكـا أسـبـابـا ونـتـائـج،
%             ثـم نـتـطرق لـعـيـوب الـعـلاجـات الـمـعـتـادة.
%             وللأعـمـال الـتـي سـبـق إنـجـازهـا فـي مـجـالـي الـتـرجـمـة الآلـية والـتـعرف الآلـي عـلى الكـلام.

%             نـأتـي بـعـدهـا إلـى تـصـمـيـم نـظام لـتـصـحـيـح الـكـلام الـمـحـتـبـس بـاللـغـة الـفـرنـسـيـة
%             يـجـمـع بـيـن نـمـوذجـيـن، أحـدهـمـا للـتـعرف الآلـي عـلى الكـلام والآخـر للـتـرجـمـة الآلـيـة
%             ونـعـرض إنـجـاز هـذا الأخـيـر.
%             نـخـتـم أخـيـرا بـعـرض نـتـائـج هـذا الـعـمـل مـتـمـثـلـة فـي مـجـمـوعـة بـيـانـات للـتـعرف الآلـي عـلى الكـلام
%             ونـمـوذج للـتـرجـمـة الآلـيـة تـقـيـيـمـه بـمـقـيـاس \LR{{bleu}} يـسـاوي \(79.61\%\).
%         \end{RLtext}
%         \hspace*{0mm}\rule{\linewidth}{1pt}
%         \begin{RLtext}
%             \textbf{الـكـلـمـات الـمـفـتـاحـيـة ـــ } حـبـسـة بـروكـا،
%             تـعـلم الآلـة،
%             مـعـالـجـة اللـغـة الـطـبـيـعـيـة،
%             تـرجـمـة آلـيـة،
%             تـعرف آلـي عـلى الكـلام،
%             شـبـكـة عـصـبـيـة غـيـر تـرتـيـبـيـة.\\
%         \end{RLtext}
%         \rule{\linewidth}{1pt}
%     \end{abstract}
%     \restoregeometry
% }