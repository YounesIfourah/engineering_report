\chapter*{General introduction}
\addcontentsline{toc}{chapter}{General introduction}
\label{chap.intro}

In the context of rapid urbanization and increasingly complex road networks, traffic monitoring has become a critical component of modern urban infrastructure. Traditional surveillance systems, often reliant on fixed sensors and costly installations, struggle to provide real-time, scalable, and cost-effective solutions. In response to these challenges, Unmanned Aerial Vehicles (UAVs), commonly known as drones, have emerged as a promising alternative for more efficient and dynamic traffic management.

This paper presents a comprehensive review of the current technologies and approaches related to the integration of UAVs in road traffic monitoring systems. It begins by examining the different UAV architectures (multirotors, fixed-wing, and hybrid configurations) along with their operational characteristics, such as weight, endurance, range, and flight altitude. The study then explores UAV-specific communication protocols (e.g., MAVLink, UranusLink, UAVCAN), assessing their strengths and limitations in terms of performance, resource efficiency, and security.

A particular focus is placed on swarm communication architectures (centralized, single-group decentralized, multi-group, and multi-layered systems) and their associated routing protocols, which significantly impact the scalability and robustness of UAV networks. Additionally, the integration of Artificial Intelligence (AI) and Machine Learning (ML) techniques is discussed as a key enabler for performance optimization, including trajectory planning, mission management, perception, and feature extraction through supervised, unsupervised, and reinforcement learning methods such as Deep Q-Networks and Deep Deterministic Policy Gradient.

Security remains a central concern, and this work identifies major threats such as eavesdropping, jamming, man-in-the-middle attacks, replay attacks, backdoors, and denial-of-service. It also outlines current countermeasures to ensure communication integrity and confidentiality.

Finally, the paper analyzes six distinct UAV-based traffic surveillance methods, ranging from Aerial Traffic Surveillance Systems (ATSS) and 5G integration to cooperative monitoring, emergency vehicle routing support, and collaborative hotspot detection. These case studies highlight the potential of UAVs to deliver real-time, adaptive, and scalable traffic management solutions.

In summary, this review underscores the transformative potential of UAVs in intelligent transportation systems while recognizing the remaining challenges in endurance, communication stability, and regulatory frameworks. Continued innovation in energy systems, network protocols, and policy development will be crucial for the widespread adoption of UAVs in future smart mobility infrastructures. This paper aims to provide a clear picture of the current technological landscape and lays the groundwork for further research into communication systems and interoperability.