\chapter*{Syntheses}

The rapid evolution of UAV technology presents unprecedented opportunities for transforming traffic surveillance and management systems. Across the three chapters, we observe how foundational advancements in UAV architectures and communication protocols enable increasingly sophisticated applications. The development of specialized communication standards and decentralized network architectures has addressed critical challenges in reliability and scalability for UAV operations. These technical foundations support the integration of AI-driven optimization techniques that enhance autonomous decision-making while maintaining robust security against emerging cyber threats.

\vspace{0.5cm}

Practical implementations demonstrate the versatility of UAV systems in traffic monitoring applications, from basic surveillance to complex cooperative frameworks. The successful deployment of 5G-enabled systems and probabilistic control models shows particular promise for real-time traffic management. However, these technological achievements must be balanced against persistent limitations in operational endurance, communication stability, and regulatory frameworks. Addressing these constraints through continued innovation in energy systems, network protocols, and policy development will determine the pace of UAV adoption in transportation infrastructure.

\vspace{0.5cm}

Looking ahead, the convergence of these technical domains - communication systems, artificial intelligence, and practical applications - points toward a future where UAV networks become integral components of smart city ecosystems. The research presented in these chapters establishes both the current state of the art and critical pathways for future development in this dynamic field. As these technologies mature, they will enable more responsive, efficient, and intelligent transportation networks capable of meeting the growing demands of urban mobility.