\vfill{}
\begin{center}
    {
        \Large\bfseries
        Resume  % Removed \RL
    }
\end{center}

Ce mémoire de fin d'études explore la mise en œuvre d'un système de communication novateur entre drones et capteurs au sol pour la surveillance du trafic routier. Face aux limitations des systèmes de surveillance traditionnels, l'intégration des Véhicules Aériens Sans Pilote (UAVs) offre des solutions prometteuses pour une gestion du trafic plus efficace.

Le document débute par une analyse approfondie des architectures des UAVs (multirotors, voilures fixes, hybrides) et de leurs caractéristiques opérationnelles (poids, endurance, portée, altitude). Il explore ensuite les protocoles de communication spécifiques aux UAVs tels que MAVLink, UranusLink et UAVCAN, en mettant en avant leurs avantages et leurs limitations, notamment en termes de sécurité et d'efficacité des ressources. Les différentes architectures de communication en essaim (centralisée, décentralisée à groupe unique, multi-groupes et multi-couches) ainsi que les protocoles de routage associés (basés sur la topologie, la position ou l'intelligence en essaim) sont également détaillés, soulignant leur impact sur la scalabilité et la robustesse.

Une partie substantielle du travail est consacrée à l'intégration de l'Intelligence Artificielle (IA) et des techniques d'apprentissage automatique (ML) pour optimiser les performances des UAVs. Cela inclut l'amélioration de la planification de trajectoire, de la gestion des missions, de la perception et de l'extraction de caractéristiques à travers des méthodes d'apprentissage supervisé, non supervisé et par renforcement (comme les Deep Q-Networks et Deep Deterministic Policy Gradient). Le mémoire aborde également les défis de sécurité cruciaux dans les réseaux d'UAVs, en identifiant les menaces (écoute clandestine, brouillage, attaques de l'homme du milieu, attaques par rejeu, portes dérobées, déni de service) et en présentant les solutions existantes pour garantir l'intégrité et la confidentialité des communications.

En conclusion, ce travail souligne que malgré les avancées significatives, des défis subsistent concernant l'endurance opérationnelle, la stabilité des communications et les cadres réglementaires. L'innovation continue dans les systèmes énergétiques, les protocoles réseau et le développement de politiques sera cruciale pour l'adoption généralisée des UAVs dans les infrastructures de transport intelligentes du futur.

\vfill{}
\pagebreak